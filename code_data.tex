% Options for packages loaded elsewhere
\PassOptionsToPackage{unicode}{hyperref}
\PassOptionsToPackage{hyphens}{url}
%
\documentclass[
]{article}
\usepackage{amsmath,amssymb}
\usepackage{lmodern}
\usepackage{ifxetex,ifluatex}
\ifnum 0\ifxetex 1\fi\ifluatex 1\fi=0 % if pdftex
  \usepackage[T1]{fontenc}
  \usepackage[utf8]{inputenc}
  \usepackage{textcomp} % provide euro and other symbols
\else % if luatex or xetex
  \usepackage{unicode-math}
  \defaultfontfeatures{Scale=MatchLowercase}
  \defaultfontfeatures[\rmfamily]{Ligatures=TeX,Scale=1}
\fi
% Use upquote if available, for straight quotes in verbatim environments
\IfFileExists{upquote.sty}{\usepackage{upquote}}{}
\IfFileExists{microtype.sty}{% use microtype if available
  \usepackage[]{microtype}
  \UseMicrotypeSet[protrusion]{basicmath} % disable protrusion for tt fonts
}{}
\makeatletter
\@ifundefined{KOMAClassName}{% if non-KOMA class
  \IfFileExists{parskip.sty}{%
    \usepackage{parskip}
  }{% else
    \setlength{\parindent}{0pt}
    \setlength{\parskip}{6pt plus 2pt minus 1pt}}
}{% if KOMA class
  \KOMAoptions{parskip=half}}
\makeatother
\usepackage{xcolor}
\IfFileExists{xurl.sty}{\usepackage{xurl}}{} % add URL line breaks if available
\IfFileExists{bookmark.sty}{\usepackage{bookmark}}{\usepackage{hyperref}}
\hypersetup{
  pdftitle={Spatial model diagnostics - overview of code and data},
  pdfauthor={Alexander Brenning},
  hidelinks,
  pdfcreator={LaTeX via pandoc}}
\urlstyle{same} % disable monospaced font for URLs
\usepackage[margin=1in]{geometry}
\usepackage{longtable,booktabs,array}
\usepackage{calc} % for calculating minipage widths
% Correct order of tables after \paragraph or \subparagraph
\usepackage{etoolbox}
\makeatletter
\patchcmd\longtable{\par}{\if@noskipsec\mbox{}\fi\par}{}{}
\makeatother
% Allow footnotes in longtable head/foot
\IfFileExists{footnotehyper.sty}{\usepackage{footnotehyper}}{\usepackage{footnote}}
\makesavenoteenv{longtable}
\usepackage{graphicx}
\makeatletter
\def\maxwidth{\ifdim\Gin@nat@width>\linewidth\linewidth\else\Gin@nat@width\fi}
\def\maxheight{\ifdim\Gin@nat@height>\textheight\textheight\else\Gin@nat@height\fi}
\makeatother
% Scale images if necessary, so that they will not overflow the page
% margins by default, and it is still possible to overwrite the defaults
% using explicit options in \includegraphics[width, height, ...]{}
\setkeys{Gin}{width=\maxwidth,height=\maxheight,keepaspectratio}
% Set default figure placement to htbp
\makeatletter
\def\fps@figure{htbp}
\makeatother
\setlength{\emergencystretch}{3em} % prevent overfull lines
\providecommand{\tightlist}{%
  \setlength{\itemsep}{0pt}\setlength{\parskip}{0pt}}
\setcounter{secnumdepth}{5}
\ifluatex
  \usepackage{selnolig}  % disable illegal ligatures
\fi

\title{Spatial model diagnostics - overview of code and data}
\author{Alexander Brenning}
\date{13 November 2021}

\begin{document}
\maketitle

\hypertarget{files}{%
\section{Files}\label{files}}

All code is written in R version 4.1.0.
CRAN or Github versions of required packages can be installed using the install script.

The following files are included in this data and code publication, using the following file naming conventions:

\begin{itemize}
\tightlist
\item
  \texttt{casestudy} is the name of the case study, i.e., \texttt{meuse} or \texttt{maipo};
\item
  \texttt{model} represents the model, for example \texttt{rf} for random forest, or \texttt{splda} for NN-LDA.
\item
  \texttt{est} represents the estimation technique, such as \texttt{loo} for (non-spatial) LOO-CV, or \texttt{spcv} for spatial CV using \(k\)-means clustering of sample coordinates.
\item
  \texttt{.rda} and \texttt{.Rdata} files were saved using \texttt{save()}, while \texttt{.rds} files used \texttt{saveRDS()}.
\end{itemize}

\emph{Code files:}

\begin{itemize}
\tightlist
\item
  \texttt{spdiag\_install.R}: Install required R packages from CRAN or Github, as appropriate.
\item
  \texttt{spdiag\_casestudy.R}: Prepare data and other objects defining, for example, model settings. Result is saved in file \texttt{run\_casestudy.rda}, in the Maipo case study also in \texttt{wrp\_maipo.rda}.
\end{itemize}

\emph{Data and results files:}

\begin{itemize}
\tightlist
\item
  The raw data of the Meuse and Maipo case studies is included in the \texttt{sp} package as \texttt{data(meuse)} and in the \texttt{sperrorest} package as \texttt{data(maipo)}, respectively.
\item
  \texttt{run\_casestudy.rda}: Preprocessed data and other object defining, for example, model settings.
\item
  \texttt{wrp\_maipo.rda}: Preprocessed data and transformation objects for model interpretation from a transformed perspective in the Maipo case study.
\item
  \texttt{casestudy\_res\_model.rda}: Results of spatial LOO estimation, to be used to compute SPEPs and SVIPs.
\item
  \texttt{casestudy\_res\_est.rda}: Results of various performance estimation procedures such as spatial CV (\texttt{est}: \texttt{spcv}) or resubstitution (\texttt{train}).
\item
  \texttt{casestudy\_smry.rds}: Summary object that contains data and (processed) results from the case studies for use in the RMarkdown manuscript file.
\end{itemize}

\hypertarget{preparing-the-data-and-computing-environment}{%
\section{Preparing the data and computing environment}\label{preparing-the-data-and-computing-environment}}

The code has been executed with R 4.1.0 and requires several additional R packages.

\begin{enumerate}
\def\labelenumi{\arabic{enumi}.}
\tightlist
\item
  Run \texttt{spdiag\_install.R}.
\item
  Run \texttt{spdiag\_casestudy.R}. You can skip this and use the pre-computed files instead.
\end{enumerate}

\hypertarget{running-the-computations}{%
\section{Running the computations}\label{running-the-computations}}

The following steps perform all computations made for this paper. Skip this section to use pre-computed results.

\begin{enumerate}
\def\labelenumi{\arabic{enumi}.}
\setcounter{enumi}{2}
\tightlist
\item
  Run each of the \texttt{run\_casestudy\_model.R} scripts.
\end{enumerate}

\begin{itemize}
\tightlist
\item
  This is the computationally most intensive part, which requires a high-performance computing environment since the code has not been optimized for performance. Note that memory requirements are high in parallel processing. Currently not all scripts make use of parallelization.
\end{itemize}

\begin{enumerate}
\def\labelenumi{\arabic{enumi}.}
\setcounter{enumi}{3}
\tightlist
\item
  Run each of the \texttt{run\_casestudy\_est.R} scripts.
\end{enumerate}

\begin{itemize}
\tightlist
\item
  This is computationally intensive especially in the Maipo case study and in the hybrid OK-RF model. Currently not all scripts make use of parallelization.
\end{itemize}

\hypertarget{summarizing-and-plotting-the-results}{%
\section{Summarizing and plotting the results}\label{summarizing-and-plotting-the-results}}

\begin{enumerate}
\def\labelenumi{\arabic{enumi}.}
\setcounter{enumi}{4}
\tightlist
\item
  Run \texttt{spdiag\_casestudy\_smry.R}.
\item
  Run \texttt{spdiag\_casestudy\_plot.R} step by step to generate the figures from the manuscript.
\end{enumerate}

\end{document}
